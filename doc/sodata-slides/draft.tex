
\section{Introduction}
\transition{
  \underline{Introduction :} \\
  \vspace{1ex}
  \large
  l'Internet, étude de la topologie de l’Internet
}

\begin{frame}{Topologie au niveau IP}
  \underline{\bf Nœuds :} ordinateurs
    
  \underline{\bf Liens :} connections IP
\end{frame}

\begin{frame}[t]{Petite note historique}

  \begin{columns}[T] % contents are top vertically aligned
    \begin{column}[T]{0.1\textwidth}
      \vspace{-1em}
      \begin{tikzpicture}
        \node (a) {};
        \node (b) [below of=a, yshift=-13em] {};
        \path[->] (a) edge node [sloped,anchor=south] {temps} (b);
      \end{tikzpicture}      
    \end{column}
    \begin{column}[T]{0.9\textwidth}
      \small
      {\color{black} 1960s --- ARPANET}
      \vspace{0.4em}
      
      {\color{black} 1980s --- Internet}
      \vspace{0.4em}

      1993 --- {\em Dynamics of internet routing information}\\
      \hfill [Chinoy]
      \vspace{0.4em}

      1997 --- \scalebox{0.85}{\em Measurements and Analysis of End-to-End Internet Dynamics}\\
      \hfill [Paxon]
      \vspace{0.4em}

      1998 --- {\em On routes and multicast trees in the Internet}\\
      \hfill [Pansiot et Grad]
      \vspace{0.4em}

      2000s --- de nombreux travaux sur la topologie  \\
      \hspace{4em} (et la dynamique) de l'Internet à grande échelle
      \vspace{0.3em}

      \hfill {\bf seulement quelques œuvres considèrent l'échelle fine}
    \end{column}
  \end{columns}
\end{frame}

